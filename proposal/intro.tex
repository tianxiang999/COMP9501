\section{Introduction}
% Sentiment analysis has become a heated topic for understanding peoples'
% opinions, attitudes, and emotions from texts (e.g., a tweet message). With the
% advance of social networks, more and more opinion-rich resources are available
% on the Internet that can help companies and the government to understand the
% public promptly and accurately. For instance, Nike posted a controversial
% advertisement campaign in 2017. Although there was a huge wave of negative
% response on twitter, sentiment analysis shows that those in favor of the
% campaign is still a large majority, so Nike kept the campaign and got a 31\%
% increment in its sales~\cite{nike1, nike2}.  

With the prevalence of cloud computing, more and more companies choose to deploy
their service in the cloud. One key advantages of deploying service on the cloud
over deploying on-premise cluster is that it can be scale-in and scale-out
easily~\cite{cloud:scale}. For instance, a web service can deploy more servers
(scale-out) during daytime to balance the loads and to handle more clients; the
service can shutdown some of the server (scale-in) in midnight.  

It is a challenging problem to making these scaling decisions in
practice. Users' workload trend and distribution is unknown and may contain 
spikes. Failing to handle these workload spikes may make server applications 
to be overloaded, or even crashed. In 2018, Amazon failed to predict the
huge amount of transactions in its Prime Day sale, which caused one of its 
storage crashed and the service to be unavailable for an hour~\cite{amazon:down}. 

Existing web development typically uses a passive scaling
method~\cite{mao2010cloud, aws:scale, azure:scale}, where a server program does
a scale-out operation if the workload exceed certain thresholds. However, this
approach has two major drawbacks. First, this threshold-based method is
sufficient for gradually-increased workloads but cannot handle workload spikes.
If the workload increases suddenly, it needs some time before a scaling-out
operation is completed. During that time, the service's Service Level Objective
(SLO) is breached. Moreover, to avoid the application from crashing and losing
data, this method needs to reject part of the user
requests~\cite{zhou2018overload}, causing bad user experience. 

Second, different from stateless applications (e.g., a web server) that can
scale out easily, scaling out a stateful application (e.g., a database or a
key-value store) takes a long time because scaling-out a stateful application
involves expensive data replications. Letting a highly loaded or overloaded
stateful application to does a scale-out operation will exacerbate its situation
and greatly increase the time for finishing the scale-out
operation~\cite{kulkarni2017rocksteady}. 


Therefore, an active method that can predict workload spikes and scaling out in
advance is highly desirable. Machine learning is a promising techniques to solve
this problem. 
% Although systems normally provides only a backend for
% user-facing applications and does not interact with clients directly,
% understanding the clients' requirement can help developers to build more
% reliable and efficient systems. Previous work shows that sentiment
% analysis can help to predict customer churns~\cite{wang2017customer,
% lella2018churn, vo2018client}, which may dues to service quality or QoS problems.
In this project, we are going to analyze the existing workload prediction
algorithms and frameworks, compare their performance, and discuss their
applicability in practice. Specifically, we will use online open source workload
traces to train our model and evaluate its performance. Moreover, we are also
considering combing workload-trace based model with sentiment analysis because
some workload spikes may come from a social event. For instance, the aforementioned Amazon
outrage was caused by the launching of a big sale.

The remaining of the proposal is organized as follows. \S 2 describes the
problem and the methods we are going to investigate. \S 3 presents the ways to
evaluate the system performance.